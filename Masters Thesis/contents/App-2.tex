\chapter{Research explorations that did not develop into viable methods}

\section{Developments that culminated in the window-based approach for the high threshold calculation}\label{sec:history_to_windows}
%This method will describe that prior to the centroid method being stumbled upon there was an in-depth exploration into establishing gradient regions in the IHH by the max. This exists in the Graphmaker script where multiple vertical lines were used to separate potential regions and a sliding window would be used to select some left-aligned region while taking the sample into account.

\section{Hysteresis Tree Method}
%This method did not pan out but was developed to estimate the high threshold by growing the structures from the brightest voxels. It was flawed in that the starting high threshold was based on the step size based on the precision as that established the sensitivity and accuracy in detrimental ways. The starting brightness was based on the lowest brightness that the top (brightest) x percent voxels have which will be brightness T. From here a binary image is generated and all voxels above T are True in the binary image. Next the brightness iterator (d) will decay from T to the low threshold with the step size being related to the precision. When iterating the rule is that voxels adjacent to True voxels with a value greater than or equal to d can be True. With this iteration the number of added True voxels per d is recorded and if this growth dips then it will be deemed as the "stopping" point. The issue is that the initial starting brightness designates possible structures and the low threshold determines what can even be bridged such that if the low threshold left a large quantity of bridging medium intensity brightness then it is viable or else it is not. Also since between steps only voxels greater or equal in brightness can be added leaving disjoints at higher precision which can cause partial collapse in the final accuracy.  